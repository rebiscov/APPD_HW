%%%%%%%%%%%%%%%%%%%%%%%%%%%%%%%%%%%%%%%%%
% Arsclassica Article
% LaTeX Template
% Version 1.1 (1/8/17)
%
% This template has been downloaded from:
% http://www.LaTeXTemplates.com
%
% Original author:
% Lorenzo Pantieri (http://www.lorenzopantieri.net) with extensive modifications by:
% Vel (vel@latextemplates.com)
%
% License:
% CC BY-NC-SA 3.0 (http://creativecommons.org/licenses/by-nc-sa/3.0/)
%
%%%%%%%%%%%%%%%%%%%%%%%%%%%%%%%%%%%%%%%%%

%----------------------------------------------------------------------------------------
%	PACKAGES AND OTHER DOCUMENT CONFIGURATIONS
%----------------------------------------------------------------------------------------

\documentclass[
10pt, % Main document font size
a4paper, % Paper type, use 'letterpaper' for US Letter paper
oneside, % One page layout (no page indentation)
%twoside, % Two page layout (page indentation for binding and different headers)
headinclude,footinclude, % Extra spacing for the header and footer
BCOR5mm, % Binding correction
]{scrartcl}

\input{structure.tex} % Include the structure.tex file which specified the document structure and layout

\hyphenation{Fortran hy-phen-ation} % Specify custom hyphenation points in words with dashes where you would like hyphenation to occur, or alternatively, don't put any dashes in a word to stop hyphenation altogether

%----------------------------------------------------------------------------------------
%	TITLE AND AUTHOR(S)
%----------------------------------------------------------------------------------------

\title{\normalfont\spacedallcaps{Article Title}} % The article title

%\subtitle{Subtitle} % Uncomment to display a subtitle

\author{\spacedlowsmallcaps{Vincent Rébiscoul}} % The article author(s) - author affiliations need to be specified in the AUTHOR AFFILIATIONS block

\date{} % An optional date to appear under the author(s)
\usepackage{caption}

%----------------------------------------------------------------------------------------

\begin{document}

%----------------------------------------------------------------------------------------
%	HEADERS
%----------------------------------------------------------------------------------------

\renewcommand{\sectionmark}[1]{\markright{\spacedlowsmallcaps{#1}}} % The header for all pages (oneside) or for even pages (twoside)
%\renewcommand{\subsectionmark}[1]{\markright{\thesubsection~#1}} % Uncomment when using the twoside option - this modifies the header on odd pages
\lehead{\mbox{\llap{\small\thepage\kern1em\color{halfgray} \vline}\color{halfgray}\hspace{0.5em}\rightmark\hfil}} % The header style

\pagestyle{scrheadings} % Enable the headers specified in this block

%----------------------------------------------------------------------------------------
%	TABLE OF CONTENTS & LISTS OF FIGURES AND TABLES
%----------------------------------------------------------------------------------------

\maketitle % Print the title/author/date block

\setcounter{tocdepth}{2} % Set the depth of the table of contents to show sections and subsections only

\tableofcontents % Print the table of contents

\listoffigures % Print the list of figures

\listoftables % Print the list of tables

%----------------------------------------------------------------------------------------
%	ABSTRACT
%----------------------------------------------------------------------------------------

\section*{Abstract} % This section will not appear in the table of
% contents due to the star (\section*)

In this report, we are going to try several configurations and with
Simgrid, we will evaluate our programs and compare the sequential
algorithms with the parralel algorithms. The topology I used for this
is a clique. The platform file and the hostfile were generated using a
modification of the script that was given to us during TPs and was
originally meant to generate a ring topology. I precise that I will
not use ring topology because I think it is not necessary (there is no
interest to run a graph algorithm on a ring). Moreover, I find it too
bad that I one link is broken, the network can not work.

%----------------------------------------------------------------------------------------

\newpage % Start the article content on the second page, remove this if you have a longer abstract that goes onto the second page

%----------------------------------------------------------------------------------------
%	INTRODUCTION
%----------------------------------------------------------------------------------------

\section{Methodology}

We are going to separate the analysis of the Kruskal's algorithm and
Prim's algorithm. To do the analysis, we will test several
cases. First I note $G=(V, E)$ our graph and $n=|V|$ and $m=|E|$. We
will test three cases for $m$:
\begin{itemize}
\item $m=2\times n$ (linear)
\item $m=\frac{n^{\frac{3}{2}}}{2}$
\end{itemize}

And I will test those configurations for $n=500$ (my
laptop can not handle bigger graphs).
I will test this with $4$ cores and $10$ cores. I will take $500$ for
the maximum weight. Indeed, this value should not have a lot of
consequences on the speed of the algorithm. I will also test a latency
of $1\mu s$ and $100\mu s$. I will call the linear graph (with
$m=2\times n$) the linear graph and the graph with $m =
\frac{n^{\frac{3}{2}}}{2}$ the $\frac{3}{2}$ graph.

\section{Results of the tests}

\subsection{Sequential algorithms}
\begin{center}
  \captionof{table}{Speed of the sequential algorithms (in seconds)}
  \begin{tabular}{lll}
    & Linear & 3/2\\
    Prim & 5$\times$ 10\(^{\text{-3}}\) & 5.3$\times$ 10\(^{\text{-3}}\)\\
    Kruskal & 2$\times$ 10\(^{\text{-3}}\) & 3.4$\times$ 10\(^{\text{-3}}\)\\
  \end{tabular}
\end{center}

\subsection{Parallel Prim's algorithm}
\begin{center}
  \captionof{table}{Speed of the parallel Prim's algorithm (in seconds)}  
  \begin{tabular}{lllll}
    & 1$\mu s$/4 cores & 1$\mu s$/8 cores & 100$\mu s$/4 cores & 100$\mu s$/8 cores\\
    Linear & 3$\times$ 10\(^{\text{-3}}\) & 4$\times$ 10\(^{\text{-3}}\) & 3$\times$ 10\(^{\text{-1}}\) & 4$\times$ 10\(^{\text{-1}}\)\\
    3/2 & 3$\times$ 10\(^{\text{-3}}\) & 4$\times$ 10\(^{\text{-3}}\) & 3$\times$ 10\(^{\text{-1}}\) & 4$\times$ 10\(^{\text{-1}}\)\\
  \end{tabular}
\end{center}

\subsection{Parallel Kruskal's algorithm}
\begin{center}
  \captionof{table}{Speed of the parallel Kruskal's algorithm (in seconds)}  
  \begin{tabular}{lllll}
    & 1$\mu s$/4 cores & 1$\mu s$/8 cores & 100$\mu s$/4 cores & 100$\mu s$/8 cores\\
    Linear & 9$\times$ 10\(^{\text{-6}}\) & 1.3$\times$ 10\(^{\text{-5}}\) & 8.6$\times$ 10\(^{\text{-4}}\) & 1.3$\times$ 10\(^{\text{-3}}\)\\
    3/2 & 9.3$\times$ 10\(^{\text{-6}}\) & 1.4$\times$ 10\(^{\text{-5}}\) & 8.5$\times$ 10\(^{\text{-4}}\) & 1.2$\times$ 10\(^{\text{-3}}\)\\
  \end{tabular}
\end{center}

First, we can see that there is little different between the linear
graph and the 3/2 graph. That could be expected because in every
cases, because the graph is represented by a matrix, the complexity
will always depends on $n^2$ and not on $m$. This is why we do not
have a lot of differences between the linear graph and the $3/2$
graph, the number of edges has little effect on the number of steps
that the algorithm will do.

Moreover, we see that when running the parallel algorithms on 8 cores,
it is a bit slower that on 4 cores. I think this comes from the fact
that my processor has only 4 cores and when running the algorithms on
8 cores, I do not computation time but I loose time in communications
between the ``virtual'' processors. Now we will compare the sequential
algorithms. I will just study the linear graph because we saw that the
results were similar with the 3/2 graph. I will only study the case
with 4 cores because this is when my computer is the more efficient.

\section{Comparison between sequential and parallel algorithms}

\subsection{Sequential Prim VS Parallel Prim}
\begin{figure}
  \centering
  \includegraphics[width=10cm]{com_prim.png}
  \caption{Comparison of sequential Prim's algorithm and parallel
    Prim's algorithm}
  \label{fig:comp_prim}
\end{figure}

As we can see on figure~\ref{fig:comp_prim} the sequential algorithm
is a little worst than the parallel one with a latency of $1\mu s$ but
a lot better than the parallel one with a latency of $100 \mu$ (please
note that the scale is logarithmic). One can conclude that the latency
has a lot of influence on the efficiency of the parallel Prim's
algorithm. This is probably due to the fact that when all the
processors have their candidates, they have to send it the processor 0
that has to handle a lot of communications at one time, and during
this time the other processor are not working.

\subsection{Sequential Kruskal VS Parallel Kruskal}

\begin{figure}
  \centering
  \includegraphics[width=10cm]{com_kruskal.png}
  \caption{Comparison of sequential Kruskal's algorithm and parallel
    Kruskal's algorithm}
  \label{fig:comp_kruskal}
\end{figure}

As we can see on figure~\ref{fig:comp_kruskal}, for a latency of $1\mu
s$, the parallel algorithm is a lot better than the sequential
one. For a latency of $100 \mu s$, the parallel algorithm is a little
better than the sequential one. First conclusion, the efficiency of
the parallel algorithm is a lot better. I think we can explain this
because in the parallel Kruskal, a processor wait little for another
one. Indeed a processor do some computations send or receive
informations and then continue working but is does not need to wait
for an answer of the processor it communicated with. Also, the
algorithm is a lot less ``centralized''. There is no processor that
have to regularly get informations from everyone, a processor only
work with one close neighbor. I think this is why the parallel
algorithm is a lot better.

\section{Conclusion}

To conclude, 

\end{document}